\begin{DoxyAuthor}{Author}
Kazuma Ishio, Univ. of Tokyo 
\end{DoxyAuthor}
\begin{DoxyDate}{Date}
Last modified on 2015/05/11
\end{DoxyDate}
\hypertarget{index_INTRO}{}\section{Introduction}\label{index_INTRO}
L\+S\+T\+D\+A\+Q is a program to perform data acquisition from L\+S\+T camera.~\newline
 The camera in a telescope consists of 265 modules called F\+E\+B. L\+S\+T\+D\+A\+Q acquires data from them through T\+C\+P/\+I\+P connection.~\newline
 At this moment minimum features are implemented.\hypertarget{index_FUNCTIONALITY_OVERVIEW}{}\subsection{Functionality Overview}\label{index_FUNCTIONALITY_OVERVIEW}

\begin{DoxyItemize}
\item T\+C\+P/\+I\+P connection to F\+E\+Bs (configurable without compilation)
\item Multi threaded data acquisition through T\+C\+P/\+I\+P (configurable without compilation)
\item Event building identifying trigger number
\item data pooling in ring buffers, to absorb inequality of sending data between F\+E\+Bs.
\end{DoxyItemize}\hypertarget{index_PROC_OVERVIEW}{}\subsection{Procedure Overview}\label{index_PROC_OVERVIEW}

\begin{DoxyItemize}
\item \hyperlink{_master_8cpp_a3c04138a5bfe5d72780bb7e82a18e627}{main()} function
\begin{DoxyItemize}
\item Sets parameters from args and sysconf.
\item Reads configuration file.
\item Creates structs for connetions.
\item Creates Builder thread.
\item Creates Collector threads.
\item Creates Monitor thread.
\item Waits for the end of the threads.
\end{DoxyItemize}
\item \hyperlink{_master_8cpp_a568c34b708bcc89579bda6142733110a}{Collector\+\_\+thread()} function
\begin{DoxyItemize}
\item Casts the input
\item Sets C\+P\+U.
\item Searches the charged \hyperlink{structs_ring_buffer}{s\+Ring\+Buffer} structs
\item Establishes T\+C\+P/\+I\+P connections with F\+E\+Bs.
\item Synchronizes threads before starting D\+A\+Q.
\item Reads data from socket~\newline
 and writes it to Ring\+Buffer.
\item Goes out the loop.
\item Thread ends.
\end{DoxyItemize}
\item \hyperlink{_master_8cpp_a435b26afa480ba18cbef3620d0eccee2}{Builder\+\_\+thread()} function
\begin{DoxyItemize}
\item Receives the first struct for connection.
\item Searches all the structs for connections
\item Sets C\+P\+U.
\item Creates output file.
\item Waits for start synchronization.
\item Reads data written in Ring Buffers by Collector threads, combining the data with identical trigger number embedded in the data.
\item Goes out the loop when read amount of one of the connections exceeded the requested amount.
\item Thread ends.
\end{DoxyItemize}
\end{DoxyItemize}\hypertarget{index_PROCEDURE_CLOSEVIEW}{}\section{Common procedure closeview}\label{index_PROCEDURE_CLOSEVIEW}
In this section, the explanation for the procedures which are commonly used are provided.\hypertarget{index_CPU_ID}{}\subsection{C\+P\+U I\+D specification}\label{index_CPU_ID}
To adopt parallel computing, assigned are C\+P\+U I\+Ds to the threads, which are divided in a unit of procedue which can run in parallel.\hypertarget{index_CPU_ID_OVERVIEW}{}\subsubsection{Overview}\label{index_CPU_ID_OVERVIEW}
main function creates multi threads. The threads consist of one Thru\+Put\+Mes\+\_\+thread, one Builder\+\_\+thread and several Collector\+\_\+thread. Collector\+\_\+thread can run as multiple threads and the configuration of multiplicity is established by reading Connection.\+conf. The number of threads can be increased without limit. But the number of C\+P\+Us is limited, therefore too many threads cause assignment of the C\+P\+Us shared by multiple threads. Unless the number of threads does not exceed the number of C\+P\+Us, assignment of threads to C\+P\+Us is done avoiding biased assignment.\hypertarget{index_CPU_ID_PROC}{}\subsubsection{Procedues}\label{index_CPU_ID_PROC}
There are {\ttfamily n\+Coll + 2} threads to be assigned, which consist of one {\ttfamily Builder\+\_\+thread}, one {\ttfamily Thru\+Put\+Mes\+\_\+thread}, and {\ttfamily n\+Coll} threads of {\ttfamily Collector\+\_\+thread}. Using residual number, C\+P\+Us are assigned to the threads evenly. Here is the relation between thread and {\ttfamily C\+P\+U\+\_\+\+I\+D}.

\begin{TabularC}{3}
\hline
\rowcolor{lightgray}{\bf Thread name }&{\bf Number of threads }&{\bf C\+P\+U I\+D  }\\\cline{1-3}
Collector\+\_\+thread &{\ttfamily n\+Coll} &{\ttfamily Cid} \% {\ttfamily Ncpu} (where {\ttfamily Cid} = 0,1,2...{\ttfamily n\+Coll}-\/1) \\\cline{1-3}
Builder\+\_\+thread &1 &{\ttfamily n\+Coll} \% {\ttfamily Ncpu} \\\cline{1-3}
Thru\+Put\+Mes\+\_\+thread&1 &{\ttfamily n\+Coll}+1 \% {\ttfamily Ncpu} \\\cline{1-3}
\end{TabularC}


The number of C\+P\+Us is inspected using {\ttfamily sysconf()} function in {\ttfamily \hyperlink{_master_8cpp_a3c04138a5bfe5d72780bb7e82a18e627}{main()}} function, and stored as {\ttfamily Ncpu} value. When a thread starts, which C\+P\+U to use is set, specifying {\ttfamily C\+P\+U\+\_\+\+I\+D} in {\ttfamily \+\_\+\+\_\+\+C\+P\+U\+\_\+\+S\+E\+T()} function.\hypertarget{index_START_SYNC}{}\subsection{Start synchronization}\label{index_START_SYNC}
When D\+A\+Q starts, all the Collector\+\_\+thread and Builder\+\_\+thread synchronize, not to waste Ring Buffer memory in a way that Collector\+\_\+thread starts filling data in Ring Buffer before Builder\+\_\+thread gets prepared to extract it.\hypertarget{index_START_SYNC_DETAILS}{}\subsubsection{Procedures}\label{index_START_SYNC_DETAILS}

\begin{DoxyItemize}
\item When Builder\+\_\+thread gets prepared, it waits all the other threads untill they are also prepared, checking if {\ttfamily init\+End} become n\+Coll (B\+L\+D\+\_\+\+S\+T\+A\+R\+T\+\_\+\+S\+Y\+N\+C process).
\item On the other hand, Collector\+\_\+threads undergo the preparation process too. In a Collector\+\_\+thread, when it has established connection with all the charged F\+E\+Bs, it increases {\ttfamily init\+End} value by 1, stops itself, and waits for Builder\+\_\+thread to call it.
\item When all Collector\+\_\+threads get prepared ({\ttfamily init\+End} = {\ttfamily n\+Coll}), Builder\+\_\+thread broadcasts the allowance of starting D\+A\+Q to all Collector\+\_\+threads.
\item Builder\+\_\+thread starts reading data from Ring\+Buffers immediately after the announcement, while a Collector\+\_\+thread starts reading data from socket and writing data to Ring\+Buffer after it notices the announcement.
\end{DoxyItemize}\hypertarget{index_START_SYNC_PTHREAD}{}\subsubsection{Implementation}\label{index_START_SYNC_PTHREAD}
This functionality is implemented using {\ttfamily mutex\+\_\+init\+Lock} as a {\ttfamily pthread\+\_\+mutex\+\_\+t} and {\ttfamily cond\+\_\+allend} as a {\ttfamily pthread\+\_\+cond\+\_\+t}.


\begin{DoxyItemize}
\item Collector\+\_\+thread~\newline
 When a Collector\+\_\+thread gets prepared for D\+A\+Q, it locks {\ttfamily mutex\+\_\+init\+Lock} ticket by {\ttfamily pthread\+\_\+mutex\+\_\+lock()} and immediately it releases it by {\ttfamily pthread\+\_\+cond\+\_\+wait()}. But this function makes it stopped until {\ttfamily pthread\+\_\+cond\+\_\+broadcast()} is issued to {\ttfamily cond\+\_\+allend}. After receiving the signal, it again tries to lock {\ttfamily mutex\+\_\+init\+Lock} which enables it to leave {\ttfamily pthread\+\_\+cond\+\_\+wait()} function. As the last step, now having the {\ttfamily mutex\+\_\+init\+Lock} ticket, it unlocks it by {\ttfamily pthread\+\_\+mutex\+\_\+unlock()} so that the other Collector\+\_\+threads can do the same step.
\item Builder\+\_\+thread~\newline
 When {\ttfamily init\+End} = {\ttfamily n\+Coll}, Builder\+\_\+thread announces the allowance of starting D\+A\+Q as follows. First, it locks {\ttfamily mutex\+\_\+init\+Lock} which is already free from any other threads, with pthread\+\_\+mutex\+\_\+lock() method. Second, it sends signal to all the threads which have issued {\ttfamily pthread\+\_\+cond\+\_\+wait()} function with {\ttfamily cond\+\_\+allend} variable. This procedure means that it announces the start of D\+A\+Q to all the Collector\+\_\+threads. At last, it unlocks {\ttfamily mutex\+\_\+init\+Lock} with {\ttfamily pthread\+\_\+mutex\+\_\+unlock()} function for the Collector\+\_\+threads to be able to leave {\ttfamily pthread\+\_\+cond\+\_\+wait()} function.
\item Thru\+Put\+Mes\+\_\+thread~\newline
 When preparation for the measurement is completed, it does the same steps as Builder\+\_\+thread. But the steps neither do nor suffer from any harm. Here is the inspection of two possible case\+:
\begin{DoxyItemize}
\item Earlier than Builder\+\_\+thread~\newline
 If Thru\+Put\+Mes\+\_\+thread precedes Builder\+\_\+thread, Thru\+Put\+Mes\+\_\+thread serves the {\ttfamily cond\+\_\+allend} signal to Collector\+\_\+threads. Builder\+\_\+thread can also take {\ttfamily mutex\+\_\+init\+Lock} ticket right after some other thread releases it. This doesn\textquotesingle{}t take much time because all the threads releases it (by {\ttfamily pthread\+\_\+mutex\+\_\+unlock()}) immediately after it locks (as awaking {\ttfamily pthread\+\_\+cond\+\_\+wait()}). When Builder\+\_\+thread takes {\ttfamily mutex\+\_\+init\+Lock} and sends signal by {\ttfamily pthread\+\_\+cond\+\_\+broadcast()}, nothing happens, because the signal was already sent and all the threads have gone to next step. The moment Builder\+\_\+thread releases {\ttfamily mutex\+\_\+init\+Lock} as the next step and starts reading Ring\+Buffers is not so much later than
\end{DoxyItemize}
\item A thread which will be added in the future
\begin{DoxyItemize}
\item Monitor thread 
\end{DoxyItemize}
\end{DoxyItemize}